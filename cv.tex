%% start of file `template.tex'.
%% Copyright 2006-2015 Xavier Danaux (xdanaux@gmail.com).
%
% This work may be distributed and/or modified under the
% conditions of the LaTeX Project Public License version 1.3c,
% available at http://www.latex-project.org/lppl/.


\documentclass[11pt,a4paper,sans]{moderncv}        % possible options include font size ('10pt', '11pt' and '12pt'), paper size ('a4paper', 'letterpaper', 'a5paper', 'legalpaper', 'executivepaper' and 'landscape') and font family ('sans' and 'roman')

% moderncv themes
\moderncvstyle{classic}                             % style options are 'casual' (default), 'classic', 'banking', 'oldstyle' and 'fancy'
\moderncvcolor{blue}                               % color options 'black', 'blue' (default), 'burgundy', 'green', 'grey', 'orange', 'purple' and 'red'
%\renewcommand{\familydefault}{\sfdefault}         % to set the default font; use '\sfdefault' for the default sans serif font, '\rmdefault' for the default roman one, or any tex font name
%\nopagenumbers{}                                  % uncomment to suppress automatic page numbering for CVs longer than one page

% character encoding
%\usepackage[utf8]{inputenc}                       % if you are not using xelatex ou lualatex, replace by the encoding you are using
%\usepackage{CJKutf8}                              % if you need to use CJK to typeset your resume in Chinese, Japanese or Korean

% adjust the page margins
\usepackage[scale=0.75, margin=0.5in]{geometry}
%\setlength{\hintscolumnwidth}{3cm}                % if you want to change the width of the column with the dates
%\setlength{\makecvtitlenamewidth}{10cm}           % for the 'classic' style, if you want to force the width allocated to your name and avoid line breaks. be careful though, the length is normally calculated to avoid any overlap with your personal info; use this at your own typographical risks...

% personal data
\name{Gilead}{Posluns}
%\title{Resumé title}                               % optional, remove / comment the line if not wanted
%\address{street and number}{postcode city}{country}% optional, remove / comment the line if not wanted; the "postcode city" and "country" arguments can be omitted or provided empty
\phone[mobile]{+1~(647)~217~8174}                   % optional, remove / comment the line if not wanted; the optional "type" of the phone can be "mobile" (default), "fixed" or "fax"
%\phone[fixed]{+2~(345)~678~901}
%\phone[fax]{+3~(456)~789~012}
\email{gil.posluns@mail.utoronto.ca}                               % optional, remove / comment the line if not wanted
%\homepage{www.johndoe.com}                         % optional, remove / comment the line if not wanted
\social[linkedin]{gilead-posluns}                        % optional, remove / comment the line if not wanted
%\social[twitter]{jdoe}                             % optional, remove / comment the line if not wanted
\social[github]{gposluns}                              % optional, remove / comment the line if not wanted
%\extrainfo{additional information}                 % optional, remove / comment the line if not wanted
%\photo[64pt][0.4pt]{picture}                       % optional, remove / comment the line if not wanted; '64pt' is the height the picture must be resized to, 0.4pt is the thickness of the frame around it (put it to 0pt for no frame) and 'picture' is the name of the picture file
%\quote{Some quote}                                 % optional, remove / comment the line if not wanted

% bibliography adjustements (only useful if you make citations in your resume, or print a list of publications using BibTeX)
%   to show numerical labels in the bibliography (default is to show no labels)
%\makeatletter\renewcommand*{\bibliographyitemlabel}{\@biblabel{\arabic{enumiv}}}\makeatother
%   to redefine the bibliography heading string ("Publications")
%\renewcommand{\refname}{Articles}

% bibliography with mutiple entries
%\usepackage{multibib}
%\newcites{book,misc}{{Books},{Others}}
%----------------------------------------------------------------------------------
%            content
%----------------------------------------------------------------------------------
\begin{document}
%\begin{CJK*}{UTF8}{gbsn}                          % to typeset your resume in Chinese using CJK
%-----       resume       ---------------------------------------------------------
\makecvtitle
\section{Research Interests}
I explore different ways to schedule tasks for parallel execution.  This is a challenging problem because scheduling constraints
are not always known when building the schedule, so a good scheduler must adapt to new information.  When building these schedulers,
we can go beyond simply adhering to constraints and use our available flexibility to seek the best schedules, which could not necessarily
have been generated statically even if all constraints were known.  My past work has been published at ISCA, a top tier computer architecture
conference.

\section{Education}
\cventry{2022--present}{PhD}{University of Toronto}{Toronto}{}{Electrical and Computer Engineering
\begin{itemize}
%\item Bell Graduate Scholarship
\item Advisor: Professor Mark Jeffrey 
\end{itemize}}
\cventry{2020--2022}{MASc}{University of Toronto}{Toronto}{\textit{}}{Electrical and Computer Engineering
\begin{itemize}
\item Thesis: A Speculative Hardware Scheduler Supporting Priority Updates
\item Advisor: Professor Mark Jeffrey 
%\item Edward S Rogers Sr. Graduate Scholarship 
\end{itemize}}
\cventry{2015-2020}{BASc in Engineering Science with Honours}{University of Toronto}{Toronto}{}{Electrical and Computer Engineering  % arguments 3 to 6 can be left empty
\begin{itemize}
\item Thesis: Extending Multi-path Execution to a Multiprocessor Context
%\item President's Entrance Scholarship
\end{itemize}}

\nocite{posluns2022scalable}
\bibliographystyle{plain}
\bibliography{pubs}                        
%\cventry{2022}{Publication}{A Scalable Architecture for Reprioritizing Ordered Parallelism}{ISCA 2022}{DOI: 10.1145/3470496.3527387}{}
%\cventry{2022}{Masters Thesis}{A Speculative Hardware Scheduler Supporting Priority Updates}{}{}{}
%\cventry{2020}{Undergraduate Thesis}{Extending Multi-path Execution to a Multiprocessor Context}{}{}{}

\section{Scholarships and Awards}
	\cventry{2022-2023}{Bell Graduate Scholarship}{Provincial Competition}{}{\$20000}{}
	\cventry{2021-2022}{Queen Elizabeth II GSST}{Provincial Competition}{}{\$15000}{}
	%\cventry{2020-2023}{Edward S Rogers Sr Graduate Scholarships}{}{}{}{\$24000}
	\cventry{2017}{NSERC Undergraduate Student Research Award}{Department Competition}{}{\$6000}{}
	\cventry{2015}{President's Entrance Scholarship}{Academic Award}{}{\$2000}{}
	\cventry{2015-2018, 2019-2020}{Dean's List}{Academic Award}{}{}{}

\section{Work Experience}
%\subsection{Vocational}
\cventry{2020-2023}{Teaching Assistant}{University of Toronto}{}{}{
\begin{itemize}
\item ESC180/ESC190 Intro to Computer Programming
\item ECE243 Computer Organization
\item ECE344/ECE353 Operating Systems
\item ECE552 Computer Architecture
\item ECE1755 Parallel Computer Architecture and Programming
\end{itemize}}
\cventry{2018-2019}{SoC Design Engineering Intern}{Intel Corporation}{Toronto}{}{Developed and maintained tools used for silicon correlation and test pattern generation%
\begin{itemize}%
\item Designed and ran ML-based analysis of silicon correlation results for previous design families to inform correlation test stamp allocation for new Agilex device family;
\item Maintained an updated internal silicon correlation and pattern generation tools:
  \begin{itemize}%
  \item Documented and automated update process for pattern generation tools;
  \item Architected major refactor of silicon correlation flow to enable new analysis types;
%    \begin{itemize}
 %   \item Sub-sub-achievement i;
    %\item Sub-sub-achievement ii;
    %\item Sub-sub-achievement iii;
    %\end{itemize}
  \end{itemize}
\item Designed and created new pattern generation tools;
  \begin{itemize}
  \item Automated previously manual pattern generation processes for some test pattern types;
  \item Improved wire coverage of previously manual pattern types by an order of magnitude;
  \end{itemize}
\end{itemize}}
%\subsection{Research}
\cventry{2017-2017}{Research Assistant}{Intelligent Sensory Microsystems Lab}{University of Toronto}{}{NSERC USRA funded position\newline{}Developed FPGA/C++ interface to send commands to and receive output from two-bucket camera sensor prototype.}
%\subsection{Miscellaneous}
%\cventry{year--year}{Job title}{Employer}{City}{}{Description}

%\section{Languages}
%\cvitemwithcomment{Language 1}{Skill level}{Comment}
%\cvitemwithcomment{Language 2}{Skill level}{Comment}
%\cvitemwithcomment{Language 3}{Skill level}{Comment}

%\section{Skills}
%\cvdoubleitem{Programming:}{C/C++, Java, Python, Perl}{FPGAs:}{Verilog}
%\cvdoubleitem{CAD Tools:}{IAR, Quartus, Vivado}{Other:}{Arduino, Matlab, etc}

%\section{Interests}
%\cvitem{hobby 1}{Description}
%\cvitem{hobby 2}{Description}
%\cvitem{hobby 3}{Description}

%\section{Extra 1}
%\cvlistitem{Item 1}
%\cvlistitem{Item 2}
%\cvlistitem{Item 3. This item is particularly long and therefore normally spans over several lines. Did you notice the indentation when the line wraps?}

%\section{Extra 2}
%\cvlistdoubleitem{Item 1}{Item 4}
%\cvlistdoubleitem{Item 2}{Item 5\cite{book1}}
%\cvlistdoubleitem{Item 3}{Item 6. Like item 3 in the single column list before, this item is particularly long to wrap over several lines.}

%\section{References}
%\begin{cvcolumns}
 % \cvcolumn{Category 1}{\begin{itemize}\item Person 1\item Person 2\item Person 3\end{itemize}}
 % \cvcolumn{Category 2}{Amongst others:\begin{itemize}\item Person 1, and\item Person 2\end{itemize}(more upon request)}
 % \cvcolumn[0.5]{All the rest \& some more}{\textit{That} person, and \textbf{those} also (all available upon request).}
%\end{cvcolumns}

% Publications from a BibTeX file without multibib
%  for numerical labels: \renewcommand{\bibliographyitemlabel}{\@biblabel{\arabic{enumiv}}}% CONSIDER MERGING WITH PREAMBLE PART
%  to redefine the heading string ("Publications"): \renewcommand{\refname}{Articles}
%\nocite{*}
%\bibliographystyle{plain}
%\bibliography{publications}                        % 'publications' is the name of a BibTeX file

% Publications from a BibTeX file using the multibib package
%\section{Publications}
%\nocitebook{book1,book2}
%\bibliographystylebook{plain}
%\bibliographybook{publications}                   % 'publications' is the name of a BibTeX file
%\nocitemisc{misc1,misc2,misc3}
%\bibliographystylemisc{plain}
%\bibliographymisc{publications}                   % 'publications' is the name of a BibTeX file

\clearpage
%-----       letter       ---------------------------------------------------------
% recipient data
%\recipient{Company Recruitment team}{Company, Inc.\\123 somestreet\\some city}
%\date{January 01, 1984}
%\opening{Dear Sir or Madam,}
%\closing{Yours faithfully,}
%\enclosure[Attached]{curriculum vit\ae{}}          % use an optional argument to use a string other than "Enclosure", or redefine \enclname
%\makelettertitle

%Lorem ipsum dolor sit amet, consectetur adipiscing elit. Duis ullamcorper neque sit amet lectus facilisis sed luctus nisl iaculis. Vivamus at neque arcu, sed tempor quam. Curabitur pharetra tincidunt tincidunt. Morbi volutpat feugiat mauris, quis tempor neque vehicula volutpat. Duis tristique justo vel massa fermentum accumsan. Mauris ante elit, feugiat vestibulum tempor eget, eleifend ac ipsum. Donec scelerisque lobortis ipsum eu vestibulum. Pellentesque vel massa at felis accumsan rhoncus.

%Suspendisse commodo, massa eu congue tincidunt, elit mauris pellentesque orci, cursus tempor odio nisl euismod augue. Aliquam adipiscing nibh ut odio sodales et pulvinar tortor laoreet. Mauris a accumsan ligula. Class aptent taciti sociosqu ad litora torquent per conubia nostra, per inceptos himenaeos. Suspendisse vulputate sem vehicula ipsum varius nec tempus dui dapibus. Phasellus et est urna, ut auctor erat. Sed tincidunt odio id odio aliquam mattis. Donec sapien nulla, feugiat eget adipiscing sit amet, lacinia ut dolor. Phasellus tincidunt, leo a fringilla consectetur, felis diam aliquam urna, vitae aliquet lectus orci nec velit. Vivamus dapibus varius blandit.

%Duis sit amet magna ante, at sodales diam. Aenean consectetur porta risus et sagittis. Ut interdum, enim varius pellentesque tincidunt, magna libero sodales tortor, ut fermentum nunc metus a ante. Vivamus odio leo, tincidunt eu luctus ut, sollicitudin sit amet metus. Nunc sed orci lectus. Ut sodales magna sed velit volutpat sit amet pulvinar diam venenatis.

%Albert Einstein discovered that $e=mc^2$ in 1905.

%\[ e=\lim_{n \to \infty} \left(1+\frac{1}{n}\right)^n \]

%\makeletterclosing

%\clearpage\end{CJK*}                              % if you are typesetting your resume in Chinese using CJK; the \clearpage is required for fancyhdr to work correctly with CJK, though it kills the page numbering by making \lastpage undefined
\end{document}


%% end of file `template.tex'.
